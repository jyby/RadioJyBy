% vim: set tw=120 fo+=t sw=2 ts=2 :
\PassOptionsToPackage{unicode=true}{hyperref} % options for packages loaded elsewhere
\PassOptionsToPackage{hyphens}{url}
%
\documentclass[12pt,letterpaper]{report}
\usepackage{version} 
\excludeversion{TODO}
\usepackage{amssymb,amsmath}
\usepackage{ifxetex,ifluatex}
\usepackage[spanish]{babel}
\usepackage{fixltx2e} % provides \textsubscript
\ifnum 0\ifxetex 1\fi\ifluatex 1\fi=0 % if pdftex
  \usepackage[T1]{fontenc}
  \usepackage[utf8]{inputenc}
  \usepackage{textcomp} % provides euro and other symbols
\else % if luatex or xelatex
  \usepackage{unicode-math}
  \defaultfontfeatures{Ligatures=TeX,Scale=MatchLowercase}
\fi
% use upquote if available, for straight quotes in verbatim environments
\IfFileExists{upquote.sty}{\usepackage{upquote}}{}
% use microtype if available
\IfFileExists{microtype.sty}{%
\usepackage[]{microtype}
\UseMicrotypeSet[protrusion]{basicmath} % disable protrusion for tt fonts
}{}
\IfFileExists{parskip.sty}{%
\usepackage{parskip}
}{% else
\setlength{\parindent}{0pt}
\setlength{\parskip}{6pt plus 2pt minus 1pt}
}
\usepackage{hyperref}
\hypersetup{
            pdfborder={0 0 0},
            breaklinks=true}
\urlstyle{same}  % don't use monospace font for urls
\setlength{\emergencystretch}{3em}  % prevent overfull lines
\providecommand{\tightlist}{%
  \setlength{\itemsep}{0pt}\setlength{\parskip}{0pt}}
\setcounter{secnumdepth}{0}
% Redefines (sub)paragraphs to behave more like sections
\ifx\paragraph\undefined\else
\let\oldparagraph\paragraph
\renewcommand{\paragraph}[1]{\oldparagraph{#1}\mbox{}}
\fi
\ifx\subparagraph\undefined\else
\let\oldsubparagraph\subparagraph
\renewcommand{\subparagraph}[1]{\oldsubparagraph{#1}\mbox{}}
\fi

\usepackage{biblatex}
\addbibresource{referencias.bib}

% set default figure placement to htbp
\makeatletter
\def\fps@figure{htbp}
\makeatother

\usepackage{graphicx}
\usepackage{fancyhdr}

\pagestyle{fancy}
% \renewcommand{\footrulewidth}{0.4pt}
\fancyhf{}
\lhead{\rightmark}
\rhead{Página \textbf{\thepage}}
% \cfoot{Universidad de Chile - Facultad de Ciencias Físicas y Matemáticas\\Departamento de Ciencias de la Computación}

\frenchspacing
\title{{\small Introducción al Trabajo de Título \\  Propuesta de tema}
\\
{\Huge 
Mensajes Personales en Gamificación:\\
Aplicación a la técnica Pomodoro.} 
}
\author{Nicolás Salas V.\\\texttt{nikosalas@gmail.com}\\\texttt{17.478.091--9}\\Tel.: \texttt{(2) 22490790} --- \texttt{+56 9 8252 6659}}
\def\thesection{\arabic{section}}

\begin{document}

\begin{figure}[t]
\includegraphics[scale=0.83]{logo.png}
%\hspace{3.5cm}
\begin{tabular}{l}
\small Universidad de Chile\\
\small Facultad de Ciencias Físicas y Matemáticas\\
\small Departamento de Ciencias de la Computación\\
\small CC6908 Introducción al Trabajo de Título\\
\small Prof: Jérémy Barbay
\vspace{2.3cm}
\end{tabular}
\end{figure}
\maketitle

\begin{abstract}
    En este trabajo se busca comparar la efectividad de incorporar mensajes personalizados (texto, audio y
    potencialmente vídeo), música y otras técnicas de \emph{gamification} para mejorar la adhesión de los usuarios a los
    principios de la metodología de manejo de trabajo \emph{pomodoro}~\cite{cirillo2009pomodoro}. El resultado del
    trabajo consistirá en dos partes: un producto permitiendo a un usuario de incorporar todas las técnicas
    implementadas a su flujo de trabajo, y  un estudio preliminar de la efectividad de estas técnicas separadas.
\end{abstract}

\tableofcontents
\newpage

\section{Motivación}\label{motivacion}

La metodología llamada ``\emph{Pomodoro}''~\cite{cirillo2009pomodoro} es una técnica de manejo de trabajo que apunta al
aumento de la productividad. En una primera instancia, se trata de dividir el trabajo en períodos de alrededor de 25
minutos para luego tomar una pausa corta que típicamente dura 5 minutos. A esta ventana de tiempo (30 minutos) se le
asigna una \emph{unidad} llamada Pomodoro.  Luego de cumplir cuatro pomodoros, la persona puede tomar un descanso más
largo de 25 ó 30 minutos. Finalmente, después de ese largo descanso, se repite el ciclo. En un nivel de aplicación mas
profundo, se pide al usuario de asignar una cantidad de unidades de tiempo a cada tarea al inicio del periodo de trabajo
(sea un día, o una semana), y de revisar la cualidad de su asignación al fin del periodo de trabajo, de maneja a mejorar
sus estimaciones futuras.

Las técnicas de \emph{gamification}\footnote{La palabra en español es ludificación, pero para preservar el nombre que la
literatura da, se usará \emph{gamification}}~\cite{deterding2011gamification} consisten en la aplicación de elementos
lúdicos, en general mas propios a juegos, a contextos dichos ``de trabajo'' con un producto útil.  Desarrollada desde
los 80s, este técnica ha encontrado muchas aplicaciones, que sea en Marketing~\cite{hamari2010game},
educación~\cite{desousa2014systematic-gamification}, trabajo~\cite{oravec2015gamification}, o
salud~\cite{pereira2014gamification-review}. Se espera que aplicar la \emph{gamification} a ciertos procesos ayuda a la
productividad y motivación de la persona, pero se ha demostrado las limitaciones de esta
estrategia~\cite{hamari2014gamification}.

Si bien existen aplicaciones que han incorporado elementos de gamification a la técnica pomodoro, hasta la fecha no se
ha explorado la incorporación de recompensas en forma de \emph{mensajes electrónicos personalizados} (de texto, audio o
video), tales como los que han sido democratizados por aplicaciones como \texttt{WhatsApp} o \texttt{Telegram}.

\paragraph{Hipótesis} El uso de técnicas de gamification (tradicionales o basadas en musica o mensajes personales),
¿podría ayudar a algunos usuarios a \emph{aprender} y \emph{seguir} la metodología de trabajo de \emph{pomodoro}? Dentro
de estas técnicas, ¿cuáles son las más potentes para la mayoría de los usuarios? ¿Cuáles son las más potentes para cada
categoría de usuarios?

\newpage
\section{Posibles soluciones}\label{posibles-soluciones}

Para estudiar el impacto de la aplicación de técnicas de gamification sobre la adhesión y la productividad de la técnica
pomodoro sobre sus usuarios, se necesita tener una aplicación que, primero, permita la aplicación de la técnica pomodoro
y provea de mensajes útiles al usuario y, segundo, que permita tomar estadísticas de su uso, de modo de poder medir el
impacto de la aplicación de las técnicas de gamification introducidas.

Dentro de aplicaciones existentes que se pueden usar para esto hay bastantes~\cite{forest-stayfocused, focus-extension,
pomodorium, teamviz}. De hecho, algunas de ellas ya incorporan elementos de gamification o elementos de estadística. Sin
embargo, no se ha demostrado seriamente la eficacia de la aplicación de elementos de gamification sobre
Pomodoro~\cite{hamari2014gamification}. Este trabajo busca discernir si dicha aplicación es positiva, negativa, o no
tiene impacto alguno.

Por una parte, medir la eficiencia de pomodoro es una tarea que aplicaciones ya realizan, aunque no de forma
sistemática. Por ejemplo, TeamViz~\cite{teamviz} muestra la cantidad de pomodoros cumplidos y permite asociarlas con
tareas, de manera de tener una estimación de tiempo por tarea.

Por otra parte, existen aplicaciones que ya incorporan elementos de gamification para aumentar la productividad de sus
usuarios, Por ejemplo existe \texttt{Habitica}~\cite{habitica} para propósitos generales,
\texttt{Pomodorium}~\cite{pomodorium} para Pomodoro o \texttt{Zombies, Run!}~\cite{zombiesrun} para deportistas.

Por último, un elemento de gamification que se propone en este trabajo es abordar la incorporación de mensajes
personales (de texto, audio o video) como recompensa por cumplir objetivos del método pomodoro. El usuario podrá leer,
ver o escuchar estos mensajes si cumple algunas metas predefinidas. Por supuesto, puede haber otros elementos de
gamification que se incorporen a esta metodología de asignación de recompensas, pero por ahora sólo se define lo que se
ha dicho anteriormente.

Una solución que se puede usar para medir la eficiencia y el apego a la técnica pomodoro es simplemente integrar
aplicaciones que resuelvan estos temas por separado. Escuchar algunos mensajes personalizados desde \texttt{WhatsApp},
usar \texttt{TeamViz} para \emph{Pomodoro} y Habitica para incorporar los elementos del juego. Si bien en papel esto
resuleve el problema planteado, es impracticable para el estudio que se requiere. La mala interconexión entre
aplicaciones generaría cambios de contexto innecesarios que disminuirían la productividad del usuario y la cantidad de
datos proporcionados podría no ser adecuada para estudiar el apego al método pomodoro o el aumento de productividad.

Finalmente, para proponer la extensibilidad de un proyecto como este a la productividad en general, se añade un módulo
de música que permitirá al usuario seleccionar listas de reproducción para momentos de descanso o de trabajo según el
período de pomodoro en que se encuentre la persona. Esto permitirá poner algunas recompensas de audio intercaladas en la
reproducción, con la hipótesis de que ello animará al usuario en su trabajo resultando también en un aumento de
productividad~\cite{hallam2002effects}.

\newpage
\section{Objetivos del proyecto}\label{objetivos-del-proyecto}

Este trabajo tiene un objetivo general y un desglose de varios objetivos específicos que ayudan a cumplir ese objetivo
general.

\subsection{Objetivo general}\label{objetivo-general}

Proveer una aplicación de apoyo al método Pomodoro que incorpore mensajes personalizados de texto, audio, video y
elementos del juego (gamification) con el objetivo de aumentar la productividad de su usuario y medir los resultados de
aumento, baja o mantención de productividad.

\subsection{Objetivos específicos}\label{objetivos-especificos}

\subsubsection{Implementación de un servicio de apoyo al método Pomodoro}\label{implementacion-de-un-servicio-de-apoyo-a-pomodoro}

El tema central de experimentación de este trabajo gira en torno a Pomodoro. Tener una aplicación que permita a su
usuario implementar el método de forma correcta y de acuerdo a sus necesidades es esencial para poder medir la
productividad de la persona.

Dentro de los hitos de desarrollo de la aplicación que involucran este objetivo se encuentran:

\begin{enumerate}\tightlist{}
  \item Reproductor minimal: Puede reproducir música y archivos de voz personales al usuario de forma intercalada. Sin
    interfaz de usuario.
  \item Pomodoro: Puede diferenciar entre música para concentrarse y música para descansar, anuncia pausas cortas,
    largas y fines de descansos.
  \item Ajuste de parámetros: Es capaz de leer parámetros de configuración de pomodoro y personales desde un archivo de
    configuración. Todavía no hay interfaz de usuario.
  \item Interfaz de usuario minimal: Se puede cambiar opciones por medio de una interfaz gráfica minimal. Los cambios
    son reflejados en el archivo de configuración.
  \item Bloqueo de estación de trabajo: La interfaz de usuario (en adelante UI) avisa sobre hitos importantes como
    pausas cortas y largas, fin de descanso. También puede bloquear la pantalla bajo autorización del usuario.
  \item Modificación del flujo: Permite las opciones de saltarse o postergar pausas por el usuario. Por supuesto esto no
    bloquea la estación de trabajo.
  \item Implementación en plataformas de escritorio: La aplicación cuenta con una UI funcional no necesariamente gráfica
    en macOS, Linux y Windows.
\end{enumerate}

\subsubsection{Implementación de gamification sobre Pomodoro}\label{implementacion-de-gamification-sobre-pomodoro}

Con la implementación de elementos de gamification sobre Pomodoro se puede empezar a medir la adhesión y apego al método
de parte de los usuarios. En un principio el elemento que se implementará tiene que ver con el permiso a leer, escuchar
o ver mensajes personales enviados al usuario, pero esto no tiene por qué ser el único elemento de gamification
implementado.

Los hitos de desarrollo que definen este objetivo son:

\begin{enumerate}\tightlist{}
  \item Generación de estadísticas: El programa detecta los pomodoros cumplidos en un día y sus fracciones, la cantidad
    de pausas saltadas, la cantidad de tiempo de trabajo que debió ser descanso y atrasos en las pausas. Estos datos se
    guardan en un archivo para conocimiento del usuario y para medir resultados.
  \item Grabado de comentarios: Los usuarios pueden intercambiar comentarios grabados por ellos mismos según categorías
    (pausas largas, cortas y fines de descanso). La aplicación tiene opción de agregar, cambiar o eliminar archivos y
    ofrece la posibilidad de cambiar de categoría las grabaciones.
\end{enumerate}

\subsubsection{Medir resutados de productividad}\label{medir-resultados-de-productividad}

Finalmente se revisa qué tanto aumenta la productividad el uso de la aplicación desarollada y se comprueba la hipótesis
del trabajo. El objetivo es proveer una medición objetiva del cambio de productividad de los usuarios y contrastarla con
las opiniones de los mismos.

Los hitos que definen este objetivo son:

\begin{enumerate}\tightlist{}
  \item Recopilación de datos: Se recopila y se agregan todos los datos que la aplicación guarda sobre la productividad
    de los usuarios.
  \item Presentación de resultados: Se contrasta estadísticamente los datos y se comprueba si la hipótesis del trabajo
    es correcta o no. Esto es, si la aplicación de gamification ayuda a los usuarios a aprender y mantener los
    principios de pomodoro y si la productividad aumenta al aplicar gamification.
\end{enumerate}

\newpage
\section{Idea general de la solución}\label{idea-general-de-la-solucion}

El trabajo consiste en una aplicación que ayuda a sus usuarios a cumplir el método de manejo de tiempo Pomodoro. La
aplicación muestra mensajes personalizados de texto, audio o video como recompensa por apego al método. Adicionalmente,
la aplicación es capaz de reproducir listas de reproducción locales de música intercaladas con mensajes de audio
dependiendo de la etapa de pomodoro en la que se encuentre el usuario.

En los objetivos específicos del proyecto se introdujo muy sucintamente los hitos del trabajo a desarrollar. En esta
sección primero se profundiza en dichas definiciones y luego se resuelven posibles problemas que puedan aparecer durante
el desarrollo del trabajo.

\subsection{Hitos}

\subsubsection{Reproductor minimal}

El producir un reproductor de audio minimal permitirá a la aplicación incorporar el elemento de gamification propuesto
en forma de audio, además de anunciar eventos relacionados a pomodoro como pausas largas y cortas y descansos.

\subsubsection{Pomodoro}

El elemento principal de Pomodoro es el manejo del tiempo. En este hito se implementan contadores de tiempo que cuando
se cumplen generan un evento notifica a su usuario que el evento ha cumplido su tiempo. Las notificaciones son de
escritorio o en modo de audio.

\subsubsection{Ajuste de parámetros}

Luego de tener pomodoro y reproducción de audio, la incorporación de parámetros permitirá a sus usuarios ajustar, por
ejemplo, los tiempos de desanso y de trabajos. Esto se hace porque el método pomodoro especifica que los tiempos de
atención no son iguales para todas las personas. Los parámetros tentativos son:

\begin{itemize}\tightlist{}
  \item Tiempo de enfoque en trabajo
  \item Tiempos de pausas cortas
  \item Tiempo de descanso largo
  \item Carpeta de música de trabajo
  \item Carpeta de música de pausa
  \item Carpeta con mensajes personalizados
\end{itemize}

\subsubsection{Interfaz de usuario minimal}

Posteriormente se empieza a elaborar una interfaz de usuario que en este hito permite a sus usuarios ajustar los
parámetros definidos anteriormente según su volición. Estos cambios se reflejan en el archivo de configuración que la
aplicación provee.

\subsubsection{Bloqueo de estación de trabajo}

Para ayudar a cumplir con la técnica pomodoro, es útil que la estación de trabajo se bloquee durante los períodos de
pausa. Según el autor del método, es mejor salir de la estación de trabajo durante las pausas para descansar la mente y
los ojos.

\subsubsection{Modoficación del flujo de trabajo}

El usuario podría querer postergar pausas o simplemente saltárselas por razones de tiempo o concentración. Esta
aplicación permite que el usuario realice estas tareas aunque no sea apropiado para el método. El uso de esta
característica de la aplicación después sirve para medir el apego a pomodoro.

\subsubsection{Generación de estadísticas}

La aplicación grabará datos estadísticos de uso y aplicación de la técnica según lo que el usuario decida hacer con la
ejecución del método. Se mide la cantidad de postergaciones de descanso, los atrasos en los descansos y cantidad de
pomodoros cumplidos por día. Estos datos servirán en el futuro para que el usuario mismo vea cómo se ha apegado a la
técnica y para medir resultados.

\subsubsection{Incorporación de gamification}

Los usuarios podrán intercambiar mensajes dirigidos de texto, audio o video que serán usados como recompensa por cumplir
objetivos (por ejemplo, hacer cuatro pomodoros). A cada mensaje se asigna una categoría y cada usuario puede grabar
nuevos mensajes que se pueden compartir entre usuarios. Los mensajes se muestran dependiendo de la categoría del mismo.
Algunas categorías son: Anuncio de pausas largas y cortas, vuelta a la estación de trabajo y recompensas. 

\subsubsection{Implementación en plataformas de escritorio}

Con el objetivo de hacer esta aplicación ubicua, será lanzada para todas las plataformas de escritorio populares:
Windows, macOS y Linux. La interfaz de usuario será minimal pero ampliamente funcional.


\subsection{Soluciones a posibles riesgos del proyecto}

Para disminuir el riesgo del proyecto, a continuación se introduce posibles soluciones a problemas que pueden, en
primera instancia, no ser triviales de resolver.


\subsubsection{Multiplataforma}\label{multiplataforma}

La multiplataforma es un tema fundamental para este proyecto. Mientras
más multiplataforma sea el proyecto, más fácil será evaluar resultados,
pues más usuarios tendrán acceso a la aplicación desarrollada.

Para disminuir el riesgo de este problema, se ocupará un lenguaje que provea como característica el ser multiplataforma.
Dos candidatos que resuelven este problema son Java (via su librería \texttt{swing}~\cite{javaswing}) y
\texttt{Node.js}~\cite{nodejs} vía sus librerías \texttt{electron}~\cite{electron} y \texttt{React\
Native}~\cite{react-native}.

La solución es simplemente usar \texttt{Node.js}. El que \texttt{React Native}
y \texttt{electron} provean una arquitectura en la que se contruye interfaces
de usuario con HTML facilita enormemente la tarea de exportar el código a
ambientes móviles (iOS/Android) mientras que \texttt{Electron} facilita la
multiplataforma en aplicaciones de escritorio (macOS/Linux/Windows).

\subsubsection{Bloqueo de estación de trabajo}\label{bloqueo-de-estacion-de-trabajo}

Podría parecer que este es un problema que no se resuelve fácilmente. Workrave es un programa licienciado bajo la GNU
General Public License v3 que resuelve este problema. Tomar la parte de este programa que resuelve este problema
disminuye la incertidumbre sobre cómo resolver este problema, pues, aunque pueda haber otras soluciones, ya existe una
forma de abordar el problema \emph{a priori}.

\newpage
\hypertarget{metodologuxeda}{%
\section{Metodología}\label{metodologuxeda}}

En este apartado se describe el trabajo a realizar para cumplir los
objetivos

\hypertarget{iterar-los-ciclos-de-desarrollo}{%
\subsection{Iterar los ciclos de
desarrollo}\label{iterar-los-ciclos-de-desarrollo}}

Se definió anteriormente 9 hitos que marcan el proceso de desarrollo de esta aplicación. En este proceso cada hito
supera en funcionalidad al anterior y por sobre todo, \textbf{es usable}. Antes de cada iteración se debe estimar el
tiempo que la implementación tomará, de modo de ajustar las horas de trabajo de la aplicación para cumplir el objetivo
final.

\hypertarget{mediciuxf3n-de-resultados-1}{%
\subsection{Medición de resultados}\label{mediciuxf3n-de-resultados-1}}

Con al menos la plataforma de escritorio de la aplicación, se prueba la aplicación con personas que tengan interés en
aplicar la técnica pomodoro de manejo de tiempo o bien estén interesadas en aplicarle elementos de gamification a su
proceso productivo. En este período, se miden los resultados completos en una semana completa de trabajo, dos veces por
separado. Se propone esta metodología para ocupar la primera semana como medida base de diagnóstico y acostumbramientos
y la segunda semana como medición de mejora sobre la técnica. Los resultados a medir son:

\begin{itemize}
  \item Cantidad de pomodoros cumplidos
  \item Tiempo de atraso en pausas
  \item Uso de elementos de gamification provistos
  \item Opinión del usuario sobre el uso de Pomodoro
\end{itemize}

El procesamiento de estos resultados da la posibilidad de estudiar la evolución en el tiempo de los usuarios en la
adhesión y el apego a pomodoro. El contraste de esas medidas con la opinión de cada usuario permitirán dar un resultado
general de cómo cambia la productividad del usuario.

Para ver cómo impacta la gamification en todo este proceso, durante las fases de medición se comenzará sin la
característica de gamification y se introducirá durante la seguda fase de pruebas. 

En términos de la muestra escogida y por asuntos de tiempo, se propone una muestra de entre 20 a 30 personas,
deseablemente de todas las edades, pero -de nuevo-, por asuntos de tiempo, el rango etáreo es solamente deseable que sea
uniforme.

\hypertarget{planteamiento-de-posibles-extensiones}{%
\subsection{Planteamiento de posibles
extensiones}\label{planteamiento-de-posibles-extensiones}}

Dependiendo del tiempo que todo lo anterior haya tomado, puede
implementarse algunas de las extensiones propuestas en el apéndice o
exportarse a alguna otra plataforma. Si esto no llegase a ser posible,
se especificará debidamente cómo se puede extender esta aplicación para
otros contextos.

\newpage
\section{Apéndice}\label{sec:apendice}

\subsection{Posibles extensiones}\label{subsec:posibles-extensiones}

Pomodoro no es el único contexto en el que este trabajo puede ser
útil, en este trabajo se ha escogido como la técnica para medir la
efectividad de aplicar elementos de gamification a esta. A
continuación se mencionan algunos ejemplos en los que esto puede
también ser útil.

\subsubsection{Deportes}

Existen algunas aplicaciones ya mencionadas~\cite{zombiesrun},
que incorporan elementos de gamification en deportes, pero se ha
dejado de lado el factor humano y la influencia de mensajes de seres
queridos sobre la persona que utiliza estas aplicaciones. Un tema de
estudio relacionado es cómo afecta una aplicación como la que este
trabajo describe sobre deportistas y si efectivamente los ayuda a
mejorar su rendimiento.

\subsubsection{Hospitales}

Las personas en situación hospitalaria también pueden ser objeto de
estudio de una aplicación como ésta, sobre todo en pacientes que de
alguna forma pueden sentirse solos. Proveer un mecanismo de
gamification combinado con mensajes personales podría mostrar un
aumento en el ánimo.

\newpage
\section{Referencias}
\printbibliography[heading=none]

\end{document}

%%% Local Variables:
%%% mode: latex
%%% TeX-master: t
%%% End:
