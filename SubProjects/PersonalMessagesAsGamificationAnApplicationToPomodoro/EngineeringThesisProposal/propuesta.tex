\PassOptionsToPackage{unicode=true}{hyperref} % options for packages loaded elsewhere
\PassOptionsToPackage{hyphens}{url}
%
\documentclass[12pt,letterpaper]{report}
\usepackage{version} 
\excludeversion{TODO}
\usepackage{amssymb,amsmath}
\usepackage{ifxetex,ifluatex}
\usepackage[spanish]{babel}
\usepackage{fixltx2e} % provides \textsubscript
\ifnum 0\ifxetex 1\fi\ifluatex 1\fi=0 % if pdftex
  \usepackage[T1]{fontenc}
  \usepackage[utf8]{inputenc}
  \usepackage{textcomp} % provides euro and other symbols
\else % if luatex or xelatex
  \usepackage{unicode-math}
  \defaultfontfeatures{Ligatures=TeX,Scale=MatchLowercase}
\fi
% use upquote if available, for straight quotes in verbatim environments
\IfFileExists{upquote.sty}{\usepackage{upquote}}{}
% use microtype if available
\IfFileExists{microtype.sty}{%
\usepackage[]{microtype}
\UseMicrotypeSet[protrusion]{basicmath} % disable protrusion for tt fonts
}{}
\IfFileExists{parskip.sty}{%
\usepackage{parskip}
}{% else
\setlength{\parindent}{0pt}
\setlength{\parskip}{6pt plus 2pt minus 1pt}
}
\usepackage{hyperref}
\hypersetup{
            pdfborder={0 0 0},
            breaklinks=true}
\urlstyle{same}  % don't use monospace font for urls
\setlength{\emergencystretch}{3em}  % prevent overfull lines
\providecommand{\tightlist}{%
  \setlength{\itemsep}{0pt}\setlength{\parskip}{0pt}}
\setcounter{secnumdepth}{0}
% Redefines (sub)paragraphs to behave more like sections
\ifx\paragraph\undefined\else
\let\oldparagraph\paragraph
\renewcommand{\paragraph}[1]{\oldparagraph{#1}\mbox{}}
\fi
\ifx\subparagraph\undefined\else
\let\oldsubparagraph\subparagraph
\renewcommand{\subparagraph}[1]{\oldsubparagraph{#1}\mbox{}}
\fi

\usepackage{biblatex}
\addbibresource{referencias.bib}

% set default figure placement to htbp
\makeatletter
\def\fps@figure{htbp}
\makeatother

\usepackage{graphicx}
\usepackage{fancyhdr}

\pagestyle{fancy}
% \renewcommand{\footrulewidth}{0.4pt}
\fancyhf{}
\lhead{\rightmark}
\rhead{Página \textbf{\thepage}}
% \cfoot{Universidad de Chile - Facultad de Ciencias Físicas y Matemáticas\\Departamento de Ciencias de la Computación}

\frenchspacing
\title{
{\small Introducción al Trabajo de Título \\  Propuesta de tema}
\\
{\Huge 
Mensajes Personales en Gamificación:\\
Applicacion a la técnica  Pomodoro.} 
}
\author{Nicolás Salas V.\\\texttt{nikosalas@gmail.com}\\\texttt{17.478.091-9}\\Tel.: \texttt{(2) 22490790} - \texttt{+56 9 8252 6659}}
\def\thesection       {\arabic{section}}

\begin{document}

\begin{figure}[t]
\includegraphics[scale=0.83]{logo.png}
%\hspace{3.5cm}
\begin{tabular}{l}
\small Universidad de Chile\\
\small Facultad de Ciencias Físicas y Matemáticas\\
\small Departamento de Ciencias de la Computación\\
\small CC6908 Introducción al Trabajo de Título\\
\small Prof: Jérémy Barbay
\vspace{2.3cm}
\end{tabular}
\end{figure}
\maketitle

\begin{abstract}
En este trabajo se busca comparar la efectividad de incorporar mensajes personalizados (texto, audio y potencialmente vídeo), musica y otras técnicas de \emph{gamification} para mejora la adhesión de los usuarios a los principios de la metodología de manejo de trabajo \emph{pomodoro}~\cite{cirillo2009pomodoro}. El resultado del trabajo consistirá en ambos un producto permitiendo a un usuario de incorporar todas las técnicas implementadas a su flujo de trabajo, y en un estudio de la efectividad de estas técnicas separadas.
\end{abstract}

\tableofcontents
\newpage

\hypertarget{motivaciuxf3n}{%
\section{Motivación}\label{motivaciuxf3n}}

La metodología llamada ``\emph{Pomodoro}''~\cite{cirillo2009pomodoro} es una técnica de manejo de trabajo apuntando al aumento de la productividad. En una primera instancia, se trata de dividir el trabajo en períodos de alrededor de 25 minutos para luego tomar una pausa corta que típicamente dura 5 minutos. A esta ventana de tiempo (30 minutos) se le asigna una \emph{unidad} llamada Pomodoro.  Luego de cumplir cuatro pomodoros, la persona puede tomar un descanso más largo de 25 ó 30 minutos. Finalmente, después de ese largo descanso, se repite el ciclo. En un nivel de aplicación mas profundo, se pide al usuario de asignar una cantidad de unidades de tiempo a cada tarea al inicio del periodo de trabajo (sea un día, o una semana), y de revisar la cualidad de su asignación al fin del periodo de trabajo, de maneja a mejorar sus estimaciones futuras.

Las técnicas de \emph{gamification}\footnote{En Español esto se diría "ludificación". En este trabajo simplemente se dirá "gamification".}~\cite{deterding2011gamification} consisten en la aplicación de elementos lúdicos, en general mas propios a juegos, a contextos dichos ``de trabajo'' con un producto útil.  Desarrollada desde los 80s, este técnica a encontrado muchas aplicaciones, que sea en Marketing, educación, trabajo, o salud.
\begin{TODO}
Nicolás: ADD references here!
\end{TODO}
Se espera que aplicar la \emph{gamification} a ciertos procesos ayuda a la productividad y motivación de la persona, pero se demostró las limitaciones de esta estrategia~\cite{hamari2014gamification}.
%
Aun que varias técnicas de gamification consideran varios tipos de ``rewards'', no se ha explorado hasta la fecha el concepto de ``rewards'' en forma de musica, o de \emph{mensajes electrónicos personalizados} (de texto, audio o vídeo), tales que los democratizados por aplicaciones como \texttt{WhatsApp} o \texttt{Telegram}.

\paragraph{Hipótesis}
El uso de técnicas de gamification (tradicionales o basadas en musica o mensajes personales) podría ayudar a algunos usuarios a \emph{aprender} y \emph{seguir} la metodología de trabajo de \emph{pomodoro}?
A dentro de estas técnicas, cuales son las mas potentes para la mayoría de los usuarios?
Cuales son las mas potentes para cada categoría de usuarios?

\newpage
\hypertarget{posibles-soluciones}{%
\section{Posibles soluciones}\label{posibles-soluciones}}

Este trabajo esta definido por varios componentes:

\begin{enumerate}
\tightlist
\item software de apoyo en el aprendizaje y la aplicación de la metodología \emph{Pomodoro};
\item (transmisión y) reproducción de \emph{mensajes personalizados} (de texto, audio o vídeo);
\item reproducción de listas de \emph{musicas} predefinidas; y 
\item otras técnicas de Gamification (e.g. puntos, estrellas, ranking, etc...).
\end{enumerate}

Hay múltiples soluciones en cada caso,
%
desde reproductores que toman música local como \emph{VLC},
\emph{Clementine}, \emph{Amarok}; 
%
aplicaciones de intercambio de mensajes personalizados como \texttt{MSN Plus},\texttt{WhatsApp},
\texttt{Telegram} y \texttt{LINE}; y
%
aplicaciones de apoyo a la aplicación de la metodología \emph{Pomodoro}. 

Las aplicaciones o plataformas que incorporan elementos del juego para realizar tareas hoy no son nada nuevo. Dependiendo del contexto, existen varias soluciones al problema de \emph{gamificar}\footnote{En español esto sería /ludificar/, pero por consistencia se deja el término en inglés} tareas. Por ejemplo existe \texttt{Habitica} para propósitos generales o \texttt{Zombies, Run!} para deportistas.

Aún así, no hay ningún producto que unifique todas estas cosas:

\begin{itemize}
\tightlist
\item
  Implemente la técnica pomodoro con elementos del juego
\item
  Permita escuchar música
\item
  Alterne mensajes  enviados por contactos (que sean de texto, de voz o de vídeo).
\end{itemize}

Una solución evidente a esto es simplemente integrar aplicaciones que
resuelvan estos temas por separado. Escuchar algunos mensajes de aliento
desde \texttt{WhatsApp}, usar Tomato Timer para Pomodoro, VLC o Spotify para
escuchar música y Habitica para incorporar los elementos del juego. Si
bien en papel esto resuleve el problema planteado, es impracticable para
el estudio que se requiere. En realizar tareas que se podrían hacer con
uno o dos clic o secuencias de teclado, se perdería tiempo en los
cambios de contexto entre aplicaciones, generando finalmente una pérdida
de productividad, que es lo que últimamente se busca mejorar.

\newpage
\hypertarget{objetivos-del-proyecto}{%
\section{Objetivos del proyecto}\label{objetivos-del-proyecto}}

Este trabajo tiene un objetivo general y un desglose de varios objetivos
específicos que ayudan a cumplir ese objetivo general.

\hypertarget{objetivo-general}{%
\subsection{Objetivo general}\label{objetivo-general}}

Proveer un servicio de música que incorpore mensajes personalizados y
elementos del juego (gamification) a la técnica pomodoro con el objetivo
de aumentar la productividad de su usuario.

\hypertarget{objetivos-especuxedficos}{%
\subsection{Objetivos específicos}\label{objetivos-especuxedficos}}

\hypertarget{obtener-un-reproductor-de-muxfasica-extensible}{%
\subsubsection{Obtener un reproductor de música
extensible}\label{obtener-un-reproductor-de-muxfasica-extensible}}

Teniendo en cuenta que toda esta aplicación gira en torno al audio, un
reproductor de audio aparece como el componente central del proyecto.
Por la inherente extensibilidad de uso de la aplicación es imperioso que
la arquitectura que presente este reproductor permita un desarrollo de
extensiones fácilmente. Si bien tomar una aplicación de reproducción ya
existente parece el camino más fácil, el comprender e indagar en el
código puede tomar parte importante del tiempo del proyecto.

\hypertarget{implementaciuxf3n-de-un-reproductor-de-muxfasica-que-incorpore-mensajes-dirigidos}{%
\subsubsection{Implementación de un reproductor de música que incorpore
mensajes
dirigidos}\label{implementaciuxf3n-de-un-reproductor-de-muxfasica-que-incorpore-mensajes-dirigidos}}

Este proyecto contempla la implementación un reproductor de música que
incluye interfaz de usuario gráfica, es capaz de intercalar mensajes
dirigidos a su usuario según elementos de gamification a definir y
muestra notificaciones relevantes de Pomodoro. A continuación se
presenta de manera muy resumida los hitos tentativos de desarrollo del
proyecto y se describen los hitos en términos de funcionalidades que
debe cumplir.

\begin{enumerate}
\tightlist
\item
  Reproductor minimal: Puede reproducir música y archivos de voz
  dirigidos al usuario de forma intercalada. Sin interfaz de usuario.
\item
  Pomodoro: Puede diferenciar entre música para concentrarse y música
  para descansar, anuncia pausas cortas y largas.
\item
  Ajuste de parámetros: Es capaz de leer parámetros desde un archivo de
  configuración. Todavía no hay interfaz de usuario.
\item
  Interfaz de usuario minimal: Se puede cambiar opciones por medio de
  una interfaz gráfica minimal. Los cambios son reflejados en el archivo
  de configuración.
\item
  Bloqueo de estación de trabajo: La interfaz de usuario (en adelante
  UI) avisa sobre hitos importantes como pausas cortas y largas, fin de
  descanso. También puede bloquear la pantalla bajo autorización del
  usuario.
\item
  Modificación del flujo: Permite las opciones de saltarse o postergar
  pausas por el usuario. Por supuesto esto no bloquea la estación de
  trabajo.
\item
  Generación de estadísticas: El programa detecta los pomodoros
  cumplidos en un día y sus fracciones, la cantidad de pausas saltadas,
  la cantidad de tiempo de trabajo que debió ser descanso y atrasos en
  las pausas. Estos datos se guardan en un archivo para conocimiento del
  usuario y para medir resultados.
\item
  Grabado de comentarios: Los usuarios pueden intercambiar comentarios
  grabados por ellos mismos según categorías (pausas largas, cortas y
  fines de descanso). La aplicación tiene opción de agregar, cambiar o
  eliminar archivos y ofrece la posibilidad de cambiar de categoría las
  grabaciones.
\item
  Incorporación de gamification: La aplicación provee elementos del
  juego que mantienen al usuario motivado. Qué elementos se incorporan
  es una tarea que este documento en particular no resuelve.
\item
  Implementación en plataformas de escritorio: La aplicación cuenta con
  una UI funcional en macOS, Linux y Windows.
\end{enumerate}

\hypertarget{medir-resultados-de-productividad}{%
\subsubsection{Medir resultados de
productividad}\label{medir-resultados-de-productividad}}

Finalmente se revisa que tanto aumenta la conveniencia o productividad
el uso del reproductor musical desarrollado. Usando los archivos
proporcionados por la aplicación se mide la evolución de la técnica
pomodoro y si ésta se ajusta a lo que el mismo usuario se propuso. Con
estos datos se tiene una evaluación objetiva de la productividad del
usuario. Para medir subjetivamente, se incorporará la opinión de cada
usuario.

\hypertarget{proveer-una-aplicaciuxf3n-fuxe1cil-de-extender}{%
\subsubsection{Proveer una aplicación fácil de
extender}\label{proveer-una-aplicaciuxf3n-fuxe1cil-de-extender}}

La planificación de este proyecto contempla sólo una extensión de varias
posibles. Al final del proyecto es objetivo especificar posibles
adiciones futuras a la aplicación.

\newpage
\hypertarget{idea-general-de-la-soluciuxf3n}{%
\section{Idea general de la
solución}\label{idea-general-de-la-soluciuxf3n}}

Para describir una idea general de solución, se hace un desglose de
soluciones de un problema por subsección.

\hypertarget{quuxe9-es-el-proyecto}{%
\subsection{Qué es el proyecto}\label{quuxe9-es-el-proyecto}}

Por obvio que parezca, no está de más explicitar que este proyecto debe
ser una aplicación de escritorio o móvil que realice las tareas
descritas en la sección de objetivos del proyecto. Con ello se hará
fácilmente computable los resultados para posterior evaluación.

\hypertarget{usar-cuxf3digo-existente-o-empezar-de-cero}{%
\subsection{Usar código existente o empezar de
cero}\label{usar-cuxf3digo-existente-o-empezar-de-cero}}

El discernir entre usar un reproductor hecho y consolidado versus
desarrollar un reproductor desde cero también representa un problema.
Usar un reproductor ya hecho introduce el problema de la internalización
de la arquitectura y código del mismo. Más aún, introduce muchas veces
el grave problema de la "multiplataforma", en el que implementar la
solución en varias plataformas es difícil. Para evitar el riesgo que
ello introduce, se desarrollará una aplicación desde cero.

\hypertarget{introducciuxf3n-de-la-gamification}{%
\subsection{Introducción de la
gamification}\label{introducciuxf3n-de-la-gamification}}

Dentro del objetivo específico de implementación, el último punto en
particular menciona que discernir qué elementos de gamification
incorporar es un tema que este \textbf{documento} no resuelve. Queda
como trabajo para el informe final de \textbf{introducción} del proyecto
discernir qué elementos se incorporarán.

\hypertarget{multiplataforma}{%
\subsection{Multiplataforma}\label{multiplataforma}}

La multiplataforma es un tema fundamental para este proyecto. Mientras
más multiplataforma sea el proyecto, más fácil será evaluar resultados,
pues más usuarios tendrán acceso a la aplicación desarrollada.

Para disminuir el riesgo de este problema, se ocupará un lenguaje que
provea como característica el ser multiplataforma. Dos candidatos que
resuelven este problema son Java (via su librería swing) y Node.js vía
sus librerías \texttt{electron}\cite{electron} y \texttt{React\ Native}\cite{react-native}.

La solución es simplemente usar Node.js. El que React Native y electron
provean una arquitectura en la que se contruye interfaces de usuario con
HTML facilita enormemente la tarea de exportar el código a ambientes
móviles (iOS/Android) mientras que Electron facilita la multiplataforma
en aplicaciones de escritorio (macOS/Linux/Windows).

\hypertarget{mediciuxf3n-de-resultados}{%
\subsection{Medición de resultados}\label{mediciuxf3n-de-resultados}}

Gracias a que la aplicación guarda por sí misma sus estadísticas de uso,
la medición de resultados se facilita simplemente recolectando estos
datos bajo consentimiento de sus usuarios.

\newpage
\hypertarget{metodologuxeda}{%
\section{Metodología}\label{metodologuxeda}}

En este apartado se describe el trabajo a realizar para cumplir los
objetivos

\hypertarget{planteamiento-de-elementos-de-gamification-a-incorporar}{%
\subsection{Planteamiento de elementos de gamification a
incorporar}\label{planteamiento-de-elementos-de-gamification-a-incorporar}}

El primer paso es decidir qué elementos de juego se va a incorporar a la
aplicación. Este ítem es lo primero porque ayuda a la planificación
general del proyecto, contribuyendo a la adaptación al tiempo total de
desarrollo que, dicho sea de paso, interesa minimizar.

\hypertarget{iterar-los-ciclos-de-desarrollo}{%
\subsection{Iterar los ciclos de
desarrollo}\label{iterar-los-ciclos-de-desarrollo}}

Se definió anteriormente 10 hitos que marcan el proceso de desarrollo de
esta aplicación. En este proceso cada hito supera en funcionalidad al
anterior y por sobre todo, \textbf{es usable}. Antes de cada iteración
se debe estimar el tiempo que la implementación tomará, de modo de
ajustar las horas de trabajo de la aplicación para cumplir el objetivo
final.

\hypertarget{mediciuxf3n-de-resultados-1}{%
\subsection{Medición de resultados}\label{mediciuxf3n-de-resultados-1}}

Con al menos la plataforma de escritorio de la aplicación, se prueba la
aplicación con personas que tengan interés en aplicar la técnica
pomodoro de manejo de tiempo o bien estén interesadas en aplicarle
elementos de gamification a su proceso productivo. En este período, se
miden los resultados completos en una semana completa de trabajo, dos
veces por separado. Se propone esta metodología para ocupar la primera
semana como medida base de diagnóstico y acostumbramientos y la segunda
semana como medición de mejora sobre la técnica.

En términos de la muestra escogida y por asuntos de tiempo, se propone
una muestra de entre 20 a 30 personas, deseablemente de todas las
edades, pero -de nuevo-, por asuntos de tiempo, el rango etáreo es
solamente deseable que sea uniforme.

\hypertarget{planteamiento-de-posibles-extensiones}{%
\subsection{Planteamiento de posibles
extensiones}\label{planteamiento-de-posibles-extensiones}}

Dependiendo del tiempo que todo lo anterior haya tomado, puede
implementarse algunas de las extensiones propuestas en el apéndice o
exportarse a alguna otra plataforma. Si esto no llegase a ser posible,
se especificará debidamente cómo se puede extender esta aplicación para
otros contextos.

\newpage
\section{Apéndice}
\label{sec:apendice}

\subsection{Posibles extensiones}
\label{subsec:posibles-extensiones}

Pomodoro no es el único contexto en el que este trabajo puede ser
útil, en este trabajo se ha escogido como la técnica para medir la
efectividad de aplicar elementos de gamification a esta. A
continuación se mencionan algunos ejemplos en los que esto puede
también ser útil.

\subsubsection{Deportes}

Existen algunas aplicaciones ya mencionadas\footnote{Zombies, Run!},
que incorporan elementos de gamification en deportes, pero se ha
dejado de lado el factor humano y la influencia de mensajes de seres
queridos sobre la persona que utiliza estas aplicaciones. Un tema de
estudio relacionado es cómo afecta una aplicación como la que este
trabajo describe sobre deportistas y si efectivamente los ayuda a
mejorar su rendimiento.

\subsubsection{Hospitales}

Las personas en situación hospitalaria también pueden ser objeto de
estudio de una aplicación como ésta, sobre todo en pacientes que de
alguna forma pueden sentirse solos. Proveer un mecanismo de
gamification combinado con mensajes dirigidos podría mostrar un
aumento en el ánimo.

\newpage
\section{Referencias}
\printbibliography[heading=none]

\end{document}

%%% Local Variables:
%%% mode: latex
%%% TeX-master: t
%%% End:
